\documentclass[12pt]{beamer}

\usepackage{hyperref}	% for referencing
\usepackage{tipa}		% for adding pipe symbol

\usepackage{amsmath}	% for math functions and matrix

\usepackage{multirow}	% for tables
\usepackage{array}      % for tables

%%%%%%%%%%%%%%%%%%%%%%%%%%%%%%%%%%%%%% for multiple cloumn with line
\usepackage{lipsum}
\usepackage{multicol}
\usepackage{color}
\setlength{\columnsep}{10pt}
\setlength{\columnseprule}{0.5pt}
\def\columnseprulecolor{\color{black}}
%%%%%%%%%%%%%%%%%%%%%%%%%%%%%%%%%%%%%%% for multiple cloumn with line

\usetheme{AnnArbor}				%setting theme for beamer class
\usecolortheme{whale,beaver}    %setting theme color for beamer class

\title{\textbf{DOCUMENTATION ON LATEX}}
\subtitle{Advanced Software Lab Assignment\\[10pt]\includegraphics[width=2cm]{img/logo.png}\\[5pt]
University of Kalyani\\[5pt]
Deep Dey\\[5pt] Roll No:03\\[5pt] \today}
%\author{}
\date{}

\begin{document}

\frame{\titlepage}

\begin{frame}[t]{\textbf{What is LATEX?}}
\begin{itemize}

\item LATEX is a document preparation system for high-quality typesetting. It is most often used for medium-to-large technical or scientific documents but it can be used for almost any form of publishing.

\item LaTeX is not a word processor! Instead, LaTeX encourages authors not to worry too much about the appearance of their documents but to concentrate on getting the right content. 

\end{itemize}
\end{frame}

\begin{frame}[t]{\textbf{Features of LATEX}}
\begin{itemize}
\item Typesetting journal articles, technical reports, books, and slide presentations.
\item Control over large documents containing sectioning, cross-references, tables and figures.
\item Typesetting of complex mathematical formulas.
\item Advanced typesetting of mathematics with AMS-LaTeX.
\item Automatic generation of bibliographies and indexes.
\item Multi-lingual typesetting.
\item Inclusion of artwork, and process or spot colour.
\item Using PostScript or Metafont fonts.
\end{itemize}
\end{frame}

\begin{frame}[fragile]{\textbf{Getting Started}}
\begin{columns}
\column{0.5\textwidth}
\begin{itemize}
\item To make documents in LaTex we use different \textbf{TAGS} and we define them by using a \textbf{Backslash(\textbackslash)}
\item The first line of code in any LaTeX document is the \textbf{document class} declaration command.
\item There are five
standard classes distributed with LATEX:
\textbf{article, report, book, letter, slides}
\end{itemize}

\column{0.5\textwidth}
\textbf{-article} for simple or short documents.\\
\textbf{-report} for small books and longer reports containing chapters.\\
\textbf{-book} for books.\\
\textbf{-letter} for letters, either business or personal.\\
\textbf{-slides} for making transparencies for projection on a screen.
\begin{verbatim}
\documentclass{class_name}
  
\documentclass{article}
for "an article document"
\end{verbatim}

\end{columns}
\end{frame}

\begin{frame}[fragile]{\textbf{Packages, Size, Margins and Spacing}}
\begin{itemize}
\item To use outside packages in LaTex we use
\begin{verbatim}
\usepackage[options ]{package }
Here, \usepackage[a4paper, margin=1in]{geometry}
\end{verbatim}
used to set margin 1 inch and page size A4
\item We define font size, number of cloumns in a page and also Page Type during documentclass declaration like-
\begin{verbatim}
\documentclass[12pt,twocolumn,letterpaper]{article}
\end{verbatim}
\item To add Vertical Spacing we use
\begin{verbatim}
\vspace{1cm}       -> Add 1 cm Vertical Spacing
\end{verbatim}
\item To add Horizontal Spacing we use
\begin{verbatim}
\hspace{1cm}       -> Add 1 cm Horizontal Spacing
\end{verbatim}
\end{itemize}
\end{frame}

\begin{frame}[fragile]{\textbf{Input File \& Title Page Structure}}
\begin{itemize}
\item LATEX input files must conform to a certain structure. 
\begin{verbatim}
\documentclass[options]{class}
     Preamble
     \begin{document}
          Document text
     \end{document}
\end{verbatim}
\item If you are using the article, report, or book class, you may want a title page for your
document. To do this, you need to tell LATEX to generate the title page with the command "\textbackslash maketitle".
\begin{verbatim}
\title{This is the Title} % provide title info
\author{My Name}          % provide author info
\date{the date}           % provide date
\maketitle                % print title page
\end{verbatim}
\end{itemize}
\end{frame}

\begin{frame}[fragile]{\textbf{Text Formatting and Alignment}}
\begin{itemize}
\item Basic Text Formatting syntax:
\begin{verbatim}
\textbf{Some Text}        % Make Some Text Bold
\textit{Some Text}        % Make Some Text Italic
\underline{Some Text}     % Underlined Some Text
\end{verbatim}
\item Text Alignment in Latex:
\begin{verbatim}
\begin{flushleft}
     Make all text left-justified written 
     within this scope
\end{flushleft}

\begin{flushright}
     Make all text Right-justified
\end{flushright}
\end{verbatim}
\end{itemize}
\end{frame}

\begin{frame}[fragile]{\textbf{Text Formatting and Alignment(Contd.)}}
\begin{itemize}
\item Text Alignment in Latex:
\begin{verbatim}
\begin{center}
     Make all text center allign written
\end{center}
            
\centering       % also used for center alignment
\textsc{text}    % make text in UPPERCASE
\large{text}     % make text larger than normal fort
\Large{text}     % make text larger than large text
\LARGE{text}     % make text larger than Large text
\huge{text}      % make text larger than LARGE text
\tiny text       % make text smaller than normal font
\texttt{Every One} % change text font style                 
\end{verbatim}
\end{itemize}
\end{frame}

\begin{frame}[fragile]{\textbf{Paragraph Formatting}}
\begin{itemize}
\item To add indentation to a Paragraph we use 
\begin{verbatim}
\indent The indent Paragraph Started
\end{verbatim}
\item To remove indentation to a Paragraph we use 
\begin{verbatim}
\noindent The non-indent Paragraph Started
\end{verbatim}
\item To add indentation to a Paragraph after a section we use 
\begin{verbatim}
\usepackage{indentfirst}
\end{verbatim}
\item To start a new line in latex we use 
\begin{verbatim}
\\[*][extra-space]
\end{verbatim}  
It has an optional argument, extra-space, that specifies how much extra vertical space is to be inserted before the next line.
\end{itemize}
\end{frame}

\begin{frame}[fragile]{\textbf{Paragraph Formatting(Contd.)}}
\begin{itemize}
\item To start a new paragraph with a line gap we use
\begin{verbatim}
\\*                it tells LaTeX not to 
                   start a new page after the line.
\end{verbatim}
\item To start from a new Page or  Line
\begin{verbatim}
\newpage                % starts from a new page
\pagebreak              % break the current page at 
                        the point of the command
\newline                % request a new line                        
\end{verbatim}
\item To draw horizontal line in a page we use
\begin{verbatim}
\rule[h]{w}{t}  % where h w and t are lengths
                upon user's choice
\hrulefill      % It draw a Horizonal line from
                left end to right end of a page
\end{verbatim}
\end{itemize}
\end{frame}

\begin{frame}[fragile]{\textbf{Section, Sub-section \& TOC}}

\begin{verbatim}
  \section{Section}  <-Make a Section
  Hello World!
  \subsection{Subsection}  <-Make a sub-section
  Structuring a document is easy!
  \subsubsection{Subsubsection}  <-Make sub section
  More text.                      of a sub-section
  \section{Another section}  <-Make another Sectioin
\end{verbatim}
\begin{itemize}
\item To show all Sections and sub sections in Table of Content we use
\begin{verbatim}
\tableofcontents
\end{verbatim}
\item To make all Sections and sub sections clickable in Table of Content we use a package
\begin{verbatim}
\usepackage{hyperref}
\end{verbatim}
\end{itemize}
\end{frame}

\begin{frame}[fragile]{\textbf{Un-ordered \& Ordered List}}
\begin{itemize}
\item Unordered List:
\begin{verbatim}
\begin{itemize}
     \item Each entries are denoted with a bullet
     \item Text in the entries may be of any length
\end{itemize}
\end{verbatim}
\item Ordered List:
\begin{verbatim}
\begin{enumerate}
     \item The labels consists of sequential numbers
     \item The numbers starts at 1
\end{enumerate}
\end{verbatim}
\end{itemize}
\end{frame}

\begin{frame}[fragile]{\textbf{Un-ordered \& Ordered List(Contd.)}}
\begin{itemize}
\item Changing the bullets of Unordered list:
\begin{verbatim}
\begin{itemize}   
     \item[$-$]            % From bullet to dash
     \item[$\ast$]         % From bullet to asterisk
     \item[$\CHARACTER$]   % Use any math character
\end{itemize}
\end{verbatim}
\item To change the number in Ordered list
\begin{verbatim}
Add \usepackage{enumitem} in your input file and
now we can use
% for Roman numbers
\begin{enumerate}[label=(\roman*)]
% for Alphabetical
\begin{enumerate}[label=\alph*)]
\end{verbatim}
\end{itemize}
\end{frame}

\begin{frame}[fragile]{\textbf{Latex Math \& Equations}}
\begin{itemize}
\item Using inline math - embed formulas in your text	
\begin{verbatim}
This formula $f(x) = x^2$ is an example.
\end{verbatim}
\item The most useful math envorinment is the equation environment:
\begin{verbatim}
\begin{equation*}
  1 + 2 = 3 
\end{equation*}   % The Output is:
\end{verbatim}
\begin{equation*}
  1 + 2 = 3 
\end{equation*}
\begin{verbatim}
       or   
\begin{equation} % This syntax add an equation no.
  1 + 2 = 3 
\end{equation}   % The Output is:
\end{verbatim}
\begin{equation}
  1 + 2 = 3 
\end{equation}
\end{itemize}
\end{frame}

\begin{frame}[fragile]{\textbf{Latex Math \& Equations(Contd.)}}
\begin{itemize}
\item Greek and Hebrew letters:
\begin{columns}
\column{0.25\textwidth}
\begin{itemize}
\item[$\alpha$]\textbackslash alpha
\item[$\beta$]\textbackslash beta
\item[$\gamma$]\textbackslash gamma
\item[$\pi$]\textbackslash pi
\item[$\Theta$]\textbackslash Theta
\end{itemize}

\column{0.25\textwidth}
\begin{itemize}
\item[$\delta$]\textbackslash delta
\item[$\Delta$]\textbackslash Delta
\item[$\rho$]\textbackslash rho
\item[$\sigma$]\textbackslash sigma
\item[$\tau$]\textbackslash tau
\end{itemize}

\column{0.25\textwidth}
\begin{itemize}
\item[$\Pi$]\textbackslash Pi
\item[$\varepsilon$]\textbackslash varepsilon
\item[$\Sigma$]\textbackslash Sigma
\end{itemize}
\end{columns}
\item When we use math syntax or matrix always include 
\begin{verbatim}
\usepackage{amsmath} package in your input file
\end{verbatim}
\item for more details about equations \href{https://drive.google.com/file/d/1_7dc3zLG99nnbX-k4bDiFqUAA9_w7kaT/view?usp=sharing}{\textbf{Please Click Here}}
\end{itemize}
\end{frame}

\begin{frame}[fragile]{\textbf{Matrix \& Determinant}}
The basic format to make matrix is
\begin{verbatim}
\begin{matrix}    % This syntax create a matrix 
Item1 & Item2\\     without braces
Item3 & Item4\\
Item5 & Item6\\
\end{matrix}

\begin{Bmatrix}   %for a matrix with curly brackets

\begin{vmatrix}    % This syntax create a determinant
Item3 & Item4\\
Item5 & Item6\\
\end{vmatrix}
\end{verbatim}
\end{frame}

\begin{frame}[fragile]{\textbf{Images \& Sub-images}}
\begin{itemize}
\item To insert a image we need to use some packages which are
\begin{verbatim}
\usepackage{graphicx}
\end{verbatim}
\item After package we need to insert image folder path
\begin{verbatim}
% Path related to the tex file containing folder path
     \graphicspath{{images/}}  
                           
%Path in Windows format                           
    \graphicspath{{c:/user/images/}}                              
\end{verbatim}
\item After insert package and image path the basic syntax to insert an image is
\begin{verbatim}
\begin{figure}
     \includegraphics[scale=1]{image_name}
\end{figure}
\end{verbatim}
\end{itemize}
\end{frame}

\begin{frame}[fragile]{\textbf{Images \& Sub-images(Contd.)}}
\begin{itemize}
\item Some addtional features:
\begin{verbatim}
\begin{figure}[h] 
     \includegraphics[scale=1, angle=45]{image_name}
     \caption{Image Caption}
\end{figure}
\end{verbatim}
\item The parameter \textbf{scale=1} determines size of the image. Here we can also use the following code to set image size:
\begin{verbatim}
\includegraphics[width=3cm, height=4cm]{image_name}
\end{verbatim}
\item The parameter \textbf{angle=45}  rotates the picture 45 degrees counter-clockwise. To rotate the picture clockwise use a negative number.
\end{itemize}
\end{frame}

\begin{frame}[fragile]{\textbf{Images \& Sub-images(Contd.)}}
\begin{itemize}
\item  \textbackslash begin$\{$figure$\}$[h], the parameter inside the brackets set the position of the figure. There is a list of parameter used to set position. They are

\begin{itemize}
\item[\textbf{H}]Places the float at precisely the location in the LATEX code. Requires the float package, though may cause problems occasionally. This is somewhat equivalent to h!.
\item[\textbf{h}]Place the float here, i.e., approximately at the same point it occurs in the source text (however, not exactly at the spot)
\item[\textbf{b}]Position at the bottom of the page.
\item[\textbf{t}]Position at the top of the page.
\end{itemize}

\item \textbackslash caption$\{$Some text$\}$ This line of code add caption to an image
\end{itemize}
\end{frame}

\begin{frame}[fragile]{\textbf{Images \& Sub-images(Contd.)}}
\begin{itemize}
\item Labels and cross-references\textbf{:}
\begin{verbatim}
\begin{figure}[h]
    \centering
    \includegraphics[width=0.25\textwidth]{mesh}
    \caption{a nice plot}
    \label{fig:mesh1}
\end{figure}

As you can see in the figure \ref{fig:mesh1},
the function grows near 0. Also, in the page 
\pageref{fig:mesh1} is the same example.
\end{verbatim}
\item There are three commands that generate cross-references in this example. They are
\end{itemize}
\end{frame}
\begin{frame}[fragile]{\textbf{Images \& Sub-images(Contd.)}}
\begin{itemize}
\item[ ] \textbf{\textbackslash label$\{$fig:mesh1$\}$} This will set a label for this figure. Since labels can be used in several types of elements within the document, it's a good practice to use a prefix, such as fig: in the example.
\item[ ] \textbf{\textbackslash ref$\{$fi:gmesh1$\}$} This command will insert the number assigned to the figure. It's automatically generated and will be updated if insert some other figure before the referenced one.
\item[ ] \textbf{\textbackslash pageref$\{$fig:mesh1$\}$} This prints out the page number where the referenced image appears.
\item[ ] ** The \textbf{\textbackslash caption$\{$ $\}$} is mandatory to reference a figure **
\item Another great characteristic in a LATEX document is the ability to automatically generate a list of figures. The syntax is
\begin{verbatim}
\listoffigures
\end{verbatim}
\end{itemize}
\end{frame}

\begin{frame}[fragile]{\textbf{Images \& Sub-images(Contd.)}}
\begin{itemize}
\item To use subfigure we are using float(subfloat). Subfloat is very useful for placing subfigures at any desired position to do so add \textbf{\textbackslash usepackage$\{$float,subfig$\}$} and the the following syntax:
\begin{verbatim}
\begin{figure}[h]
\centering  % Make images center align
\subfloat[Cation of Image_1]
{\includegraphics[width=0.4\textwidth, height=40mm]
{Image_name_1}\label{fig:one}}
    \hfill	% This is to fill the space horizontally.
\subfloat[Cation of Image_2]
{\includegraphics[width=0.4\textwidth, height=40mm]
{Image_name_2}\label{fig:three}}
\caption{Caption for whole image segments}
\end{figure}
\end{verbatim}
\end{itemize}
\end{frame}

\begin{frame}[fragile]{\textbf{Tables in Latex}}
\begin{columns}
\column{0.5\textwidth}
\textbf{We are creating a simple table:}
\begin{verbatim}
\begin{center}
\begin{table}[h]
\begin{tabular}{ |c|c|c| } 
 \hline
 cell1 & cell2 & cell3 \\ 
 cell4 & cell5 & cell6 \\ 
 cell7 & cell8 & cell9 \\ 
 \hline
\end{tabular}
\caption{Some Text}
\label{table:1}
\end{table}
\end{center}
\end{verbatim}

\column{0.5\textwidth}
\textbf{$\{$\ \textpipe \ c \textpipe  \ c \textpipe \ c \textpipe \ $\}$} This declares that three columns, separated by a vertical line, are going to be used in the table. Each c means that the contents of the column will be centred, you can also use \textbf{r} to align the text to the right and \textbf{l} for left alignment.\\
\centering
\textbf{Output:}
\begin{center}
\begin{table}[h]
\begin{tabular}{ |c|c|c| } 
 \hline
 cell1 & cell2 & cell3 \\ 
 cell4 & cell5 & cell6 \\ 
 cell7 & cell8 & cell9 \\ 
 \hline
\end{tabular}
\caption{Some Text}
\end{table}
\end{center}
\end{columns}
\end{frame}

\begin{frame}[fragile]{\textbf{Tables in Latex(Contd.)}}
\textbf{\textbackslash hline} This will insert a horizontal line on top of the table and at the bottom too. There is no restriction on the number of times you can use \textbackslash hline.\\[5pt]

\textbf{cell1 \& cell2 \& cell3 \textbackslash \textbackslash} Each \textbf{\&} is a cell separator and the double-backslash \textbackslash \textbackslash \ sets the end of this row.\\[5pt]

\textbf{\textbackslash caption$\{$Some Text$\}$} This will add a caption to the table same as image caption.\\[5pt]

\textbf{\textbackslash label$\{$table:1$\}$} and \textbf{\textbackslash ref$\{$table:1$\}$} They also work same as image label and image reference\\[5pt]

\textbf{\textbackslash begin$\{$table$\}$[h]}, the parameter inside the brackets set the position of the table also same as image position. There is a list of parameter used to set position which are also same as image position options.
\end{frame}

\begin{frame}[fragile]{\textbf{Tables in Latex(Contd.)}}
\textbf{When formatting a table a fixed length either for each column also can be set by following code:}
\begin{verbatim}
\begin{center}
\begin{tabular}{ | m{2cm} | m{1cm}| m{1cm} | } 
\hline
cell1 dummy text dummy text & cell2 & cell3 \\ 
\hline
cell1 dummy text dummy text & cell5 & cell6 \\ 
\hline
cell7 & cell8 & cell9 \\ 
\hline
\end{tabular}
\end{center}
\end{verbatim}
\end{frame}

\begin{frame}[fragile]{\textbf{Tables in Latex(Contd.)}}
\begin{columns}
\column{0.5\textwidth}
To set column width we need to add a package \textbackslash usepackage$\{$array$\}$\\[1cm]

The parameter \textbf{m$\{$2cm$\}$} sets a length of 2cm for first column (1cm for the other two) and centres the text in the middle of the cell.\\[5pt]
The aligning options are \textbf{m} for middle, \textbf{p} for top and \textbf{b} for bottom.

\column{0.5\textwidth}
\centering
\textbf{Output:}
\begin{center}
\begin{tabular}{ | m{2cm} | m{1cm}| m{1cm} | } 
\hline
cell1 dummy text dummy text & cell2 & cell3 \\ 
\hline
cell1 dummy text dummy text & cell5 & cell6 \\ 
\hline
cell7 & cell8 & cell9 \\ 
\hline
\end{tabular}
\end{center}
\end{columns}
\end{frame}

\begin{frame}[fragile]{\textbf{Tables in Latex(Contd.)}}
\textbf{Multi-column and multi-row cells in LaTeX tables}\\[2mm]
To use Multi-column and multi-row we need to use a package \textbf{\textbackslash usepackage$\{$multirow$\}$}\\[5mm]
\textbf{Basic commands:}
\begin{verbatim}
%multi-column
\multicolumn{number cols}{align}{text} % align: l,c,r

%multi-row
\multirow{number rows}{width}{text}

%Using * as width in the multirow command, the text 
argument’s natural width is used.
\end{verbatim}
\end{frame}
\begin{frame}[fragile]{\textbf{Tables in Latex(Contd.)}}
\textbf{Example of Multi-column and multi-row:}
\begin{verbatim}
\begin{table}[h]
\begin{tabular}{|c|c|c|c|c|}
\hline
\multicolumn{5}{|c|}{Multi-Column}             \\
                           \hline
\multirow{3}{*}{Multi-Row} & 7  & 8  & 9  & 10 \\ 
                           \cline{2-5} 
                           & 12 & 13 & 14 & 15 \\
                           \cline{2-5} 
                           & 17 & 18 & 19 & 20 \\
                           \hline
\end{tabular}
\end{table}
\end{verbatim}
\end{frame}

\begin{frame}[fragile]{\textbf{Tables in Latex(Contd.)}}
\begin{columns}
\column{0.5\textwidth}
\begin{verbatim}
\multicolumn{5}{|c|}{Multi-Column} % To merge 5 columns

% To merge 3 rows as output
\multirow{3}{*}{Multi-Row}
\end{verbatim}

\textbf{\textbackslash cline$\{$2-5$\}$} is used to draw a horizontal line from 
column no. 2 to column no. 5 (starting index 1).\\
Actually \textbackslash cline$\{$ $\}$ command used to draw horizontal line within a fixed column range.\\

• To create a table in landscape we used \textbf{\textbackslash begin$\{$sidewaystable$\}$} instead of \textbf{\textbackslash begin$\{$table$\}$}

\column{0.5\textwidth}
\centering
\textbf{Output:}
\begin{center}
\begin{table}[h]
\begin{tabular}{|c|c|c|c|c|}
\hline
\multicolumn{5}{|c|}{Multi-Column}             \\ \hline
\multirow{3}{*}{Multi-Row} & 7  & 8  & 9  & 10 \\ \cline{2-5} 
                           & 12 & 13 & 14 & 15 \\ \cline{2-5} 
                           & 17 & 18 & 19 & 20 \\ \hline
\end{tabular}
\end{table}
\end{center}
\centering
\ \ 
\textbf{**} Finally to create a list of tables we use \textbf{\textbackslash listoftables} command \textbf{**}
\end{columns}

\end{frame}

\begin{frame}[fragile]{\textbf{Bibliography in Latex}}
\begin{itemize}
\item We are using Bibtex for Bibliography in Latex.
\item When using Bibtex, the bibliography style is set and the bibliography file is imported with the following two commands:
\begin{verbatim}
\bibliographystyle{stylename}
\bibliography{bibfile}
\end{verbatim}
where \textbf{bibfile} is the name of the bibliography \textbf{.bib} file, without the extension and \textbf{stylename} is one of the following:
\begin{itemize}
\item abbrv
\item acm
\item plain and etc. 
\end{itemize}
\item This \textbackslash cite$\{$label$\}$ is used insert a citation where label is the label of a bibliographic entry in a .bib file depends on the bibliography style used. 
\end{itemize}
\end{frame}

\begin{frame} [fragile]{\textbf{Bibliography in Latex(Contd.)}}
\begin{itemize}
\item Bibliographic references are usually kept in a bibliography file(.bib), this file consists of a list of records and fields. Each bibliography record holds relevant information for a single entry.\\
\textbf{Syntax for .bib file:}
\begin{verbatim}
@article{einstein,
    author =   "Albert Einstein",
    title =    "{Zur Elektrodynamik bewegter 
               K{\"o}rper}. ({German})
    [{On} the electrodynamics of moving bodies]",
    journal =  "Annalen der Physik",
    volume =   "322",    number =   "10",
    pages =    "891--921",
    year =     "1905",
    DOI =      "http://dx.doi.org/10.1002" }
\end{verbatim}
\end{itemize}
\end{frame}

\begin{frame}[fragile]{\textbf{Bibliography in Latex(Contd.)}}
\textbf{@article$\{$...$\}$} \ This is the first line of a record entry, @article denotes the entry type and tells BibTeX that the information stored here is about an article. There are a lot more types like \textbf{book, conference, unpublished, misc, manual and etc}.\\[0.2cm]
\textbf{einstein} \ The label einstein is assigned to this entry, is an identifier that can be used to refer this article within the document.\\[0.2cm]
\textbf{author =   "Albert Einstein",} \ This is the first field in the bibliography entry, indicates that the author of this article is Albert Einstein. Several comma-separated fields can be added using the same syntax key = value, for instance: \textbf{title, pages, year, URL, volume, ISBN, journal, edition, chapter, copyright, etc.} 
\end{frame}

\begin{frame}[fragile]{\textbf{Bibliography in Latex(Contd.)}}
\begin{itemize}
\item Remember if you do not refer a particular entry in your bibliography it will note be shown in reference.
\item  To make sure bibliography is present in table of contents add this following line of syntax right before bibliography style
\begin{verbatim}
\addcontentsline{toc}{section}{References}
\bibliographystyle{plain}
\bibliography{filename without .bib}
\end{verbatim}
\end{itemize}
\end{frame}

\begin{frame}[fragile]{\textbf{Chapter \& Page Style}}
\begin{itemize}
\item Basically \textbackslash chapter is used in \textbf{report} and \textbf{book} class.\\ The syntax is:
\begin{verbatim}
\chapter{chapter_name}
\end{verbatim}
\item To customize a chapter we can use a package:
\begin{verbatim}
\usepackage[style_name]{fncychap}
\end{verbatim}
where \textbf{style\_name} can be \textbf{Sonny, Lenny, Glenn, Conny, Rejne, Bjarne, and Bjornstrup} depending upon user's choice.
\item The \textbackslash pagestyle command changes the style from the current page on throughout the remainder of your document. The valid options are:
\begin{itemize}
\item \textbf{headings} \ Puts running headings on each page. The document style specifies what goes in the headings.
\item \textbf{plain} \ Just a plain page number.
\item \textbf{empty} \ Produces empty heads and feet - no page numbers.
\end{itemize}
\end{itemize}
\end{frame}

\begin{frame}[fragile]{\textbf{Chapter \& Page Style(Contd.)}}
\begin{itemize}
\item we can also customize the header footer using \textbackslash pagestyle. So the syntax is:
\begin{verbatim}
\usepackage{fancyhdr} % Add this package
\pagestyle{fancy}     % make style name "fancy"
\fancyhf{}            
\rhead{text}          % add text in left header
\lhead{text}          % add text in right header
\rfoot{\thepage}      % add page no. in right footer
\lfoot{text}          % add text in right footer
\end{verbatim}
\end{itemize}
\end{frame}

\begin{frame}[fragile]{\textbf{Multi-column in Latex}}
\begin{itemize}
\item We can also use multiple column in a LATEX document by using the following code:
\begin{verbatim}
\usepackage{lipsum}   %add this package for dummytext
\usepackage{multicol} %add this package for multi-row
\usepackage{color}    %add this package for colour
\setlength{\columnsep}{10pt} %add hspace between col
\setlength{\columnseprule}{0.5pt} %set line width
\def\columnseprulecolor{\color{black}}%set line color

\begin{document}
    \begin{multicols}{2}  %make tow columns
        \lipsum[1-3]      %add some dummy text
    \end{multicols}
\end{document}
\end{verbatim}
\end{itemize}
\end{frame}

\begin{frame}[fragile]{\textbf{Multi-column in Latex(Contd.)}}
\textbf{Output:}
\begin{tiny}
\begin{multicols}{2}
\lipsum[1-3]
\end{multicols}
\end{tiny}
\end{frame}

\begin{frame}[fragile]{\textbf{Beamer class(Introduction)}}
\begin{itemize}
\item \textbf{beamer} is one of the document class which is used to create slides. Basic structure of beamer class is:
\begin{verbatim}
\documentclass[12pt]{beamer} %set the document class
\title{Some text}          %add title
\subtitle{some text}       %add sub-title
\author{author name}       %add author name
\institute{institute name} %add institute name
\date{date(optional)}      %add date

\begin{frame}         %create a single slide
\frametitle{text}     %heading of the slide
\framesubtitle{text}  %sub-heading of the slide
\titlepage            %print the title in a slide
\end{frame}           %end the created slide
\end{verbatim}
\end{itemize}
\end{frame}

\begin{frame}[fragile]{\textbf{Beamer class(Themes)}}
\begin{itemize}
\item In beamer class we can also use custom theme for slide.The basic syntax is:
\begin{verbatim}
\usetheme{AnnArbor}           %setting theme
\usecolortheme{whale,beaver}  %setting theme color
\end{verbatim}
\item To know more about themes visit: \href{https://deic-web.uab.cat/~iblanes/beamer_gallery/index.html}{\textbf{Beamer Theme Gallery}}
\end{itemize}
\end{frame}

\begin{frame}[fragile]{\textbf{Beamer class(Pause)}}
The \textbf{pause} command tells Beamer to make a new PDF page for what follows. This can be done basically anywhere.\\
\textbf{Latex Syntax for pause:}
\begin{verbatim}
\begin{frame}

Some content
 \pause
 Some more content
\end{verbatim}
\textbackslash end$\{$frame$\}$\\[0.2in]
In the above command it will create 2 slides where 1st slide print only some content but the 2nd slide will show some content with Some more content that increase code reusability.
\end{frame}

\begin{frame}[fragile]{\textbf{Beamer class(Table of Contents)}}
\begin{itemize}
\item To add table of content we use the following command:
\begin{verbatim}
\begin{frame}
\frametitle{Table of Contents}
\tableofcontents
\end{verbatim} \textbackslash end$\{$frame$\}$\\[0.2in]
\item Remember one thing \textbf{\textbackslash tableofcontents} only shows sections and sub-sections.
\end{itemize}
\end{frame}

\begin{frame}[fragile]{\textbf{Beamer class(Table of Contents(Contd.))}}
\begin{itemize}
\item It's also possible to put the table of contents at the beginning of each section and highlight the title of the current section. Just add the code below:
\begin{verbatim}
\AtBeginSection[]{
\begin{frame}
    \frametitle{Table of Contents}
    \tableofcontents[currentsection]
\end{verbatim}\textbackslash end$\{$frame$\}\}$
\item If you use \textbackslash AtBeginSubsection[] instead of \textbackslash AtBeginSection[], the table of contents will appear at the beginning of each subsection.
\item And other features of beamer class is pretty much same as other document class.
\end{itemize}
\end{frame}
\end{document}